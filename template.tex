\documentclass{article}

\usepackage{polski}
\usepackage[margin=1in]{geometry}
\usepackage[parfill]{parskip}
\usepackage{amsmath}
\usepackage{graphicx}
\usepackage{caption}
\usepackage{siunitx}
\usepackage[table,xcdraw]{xcolor}
\usepackage{hyperref}
\usepackage[T1]{fontenc}
\usepackage{comment}
\usepackage{float}

\begin{document}

\begin{center}
\bgroup
\def\arraystretch{1.5}
\begin{tabular}{|c|c|c|c|c|c|}
	\hline
	EAIiIB & \multicolumn{2}{|c|}{\begin{tabular}{@{}c@{}}Stanisław Borowy \\ Maciej Bobrek \end{tabular}} & Rok II & Grupa 1 & Zespół 5 \\
	\hline
	\multicolumn{3}{|c|}{\begin{tabular}{c}Temat: \\ \end{tabular}} & 
	\multicolumn{3}{|c|}{\begin{tabular}{c}Numer ćwiczenia: \\ \end{tabular}} \\
	\hline
	\begin{tabular}{@{}c@{}}Data wykonania \\ \end{tabular} & \begin{tabular}{@{}c@{}}Data oddania \\ \end{tabular} & 
	\begin{tabular}{c}Zwrot do popr.\\\phantom{data} \end{tabular} & \begin{tabular}{c}Data oddania\\\phantom{data}\end{tabular} &
	\begin{tabular}{c}Data zaliczenia\\\phantom{data}\end{tabular} & \begin{tabular}{c}Ocena\\\phantom{ocena}\end{tabular} \\[4ex]
	\hline
\end{tabular}
\egroup
\end{center}
\end{document}
